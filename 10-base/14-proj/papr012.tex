% file: xxxx.tex (tpl-41)
% -
% Copyright - Glen Ritchie 2025
% License terms: MIT Free for template
%\documentclass[14pt]{extarticle}
\documentclass[12pt]{article}
\usepackage[small,sf,bf]{titlesec}%0pt
\usepackage{amsmath,amssymb,amsfonts,latexsym}
\usepackage[tmargin=1.0in,bmargin=1.0in,lmargin=1.0in,rmargin=1.0in]{geometry}
% Going with 'extarticle' and 14 point, for reading script purposes.

% ============================================================
\usepackage{setspace,graphics,rotating,amsthm}
\usepackage{lipsum,hyperref,hyperxmp,graphicx}
% csquotes conflicts with 'graphicx', if it is used?
% graphicx best for PDF import with XeLaTeX or EPS with latex only build program.
\usepackage{enumerate,xspace,calc,pifont,caption}

\usepackage{tikz,pgfplots,pgfplotstable}
\usepackage{fancyhdr}
\usepackage{fontspec}
\setmainfont{Times New Roman}

\newcommand*\aakeywa{%
Writing% <-- problem
}
\newcommand{\aatita}{The Pre-Pub Paper}%
\newcommand{\aaauta}{Glen Ritchie}%

\hypersetup{
    pdftitle    = {\aatita},
    pdfauthor   = {\aaauta},
    pdfsubject  = {Paper on Number Theory \& number triangles},
    pdfkeywords = {Number Theory, number triangles, math}
}

\author{\aaauta}
\title{\aatita}
\newcommand{\thesubtitle}{Undefined}

% ============================================================
\usepackage{cleveref}% Important - to be last one - called.
\usepackage{clever-ref}

% ============================================================
% (switch to other template!) \usepackage{euler-sym}
\usepackage{plain-sym}
\usepackage{writeB-macs}

\usepackage{pdfpages}

%\let\defpgsection\section
%\renewcommand{\section}{\newpage\defpgsection}%
% If want each section, on it's own page.

\newcommand{\aadots}{\dots\,}%
\newcommand{\aabart}{---\,\,}%

% Try to avoid sfrac from xfrac package.
\newcommand{\slfrac}[2]{\,^{#1}\!/_{#2}}

\newcommand{\aacenterrule}{%
  \begin{center}
    \rule{1in}{0.4pt}%
  \end{center}
}

% \setstretch{0.95}
% \setstretch{1.00}
%\doublespacing
\setlength{\parindent}{0pt}
\setlength{\parskip}{10.0pt}

\pagestyle{fancy}
\fancyhf{}              % clear all header and footer fields
%\fancyhead[C]{\thepage} % page number in upper right
\renewcommand{\headrulewidth}{0pt} % optional: remove header line
\renewcommand{\footrulewidth}{0pt} % optional: remove footer line
\fancyfoot[C]{\thepage}

% ============================================================

\begin{document}
 %\begin{abstract}
 %
 % I don't have an abstract. This is just my paper and there you go.
 %
 %\end{abstract}
 %\pagestyle{empty}
 
 \pagenumbering{gobble}

 \maketitle   
 
 %\tableofcontents
 \newpage
 \pagenumbering{arabic}
 
    \section{Presentation}
    \label{book:01}
 
     % ------------------------------------------------------------------
 
    \subsection{Preface}
    \label{sec:10}
 
     My name is Glen Ritchie. I have been a lifelong learner and student of Mathematics as a 
     hobby.
     
     I graduated from NDSU (2002) with a Bachelor of Science in Computer Science.
     
     I am writing this to explain the many number triangles that I have found. Plus, there is a 
     Number Triangle application written in Lazarus, a Pascal-derived language.
     
     For more on the NTA2 application. See it's section.
     
     So why study number triangles? I've been playing with Number Theory for decades. It's been 
     one of my primary interests. For instance, I wrote a paper on Pythagorean triples, which is 
     available online.
     
     Yet, my work on number triangles is novel.
     
     Many people have enabled me to undertake this work, and I give glory to God for it. I thank 
     my Lord Jesus.
     
     Sincerely, Glen Ritchie
     
     Bismarck, North Dakota - 2025

    \aacenterrule
 
     % ------------------------------------------------------------------
 
    \subsection{Introduction}
    \label{sec:11}

    The subject of my presentation is number triangles.

    Why study number triangles? The Binomial numbers and their patterns in Pascal's Triangle are 
    the answer. (Or Binomial coefficients.)
    
    Its equation is: (fill-in)
    
    Pascal's Triangle's addition rule with Binomial numbers is:
    
    (fill-in)
    
    How do you compute the Binomial numbers? Answer: It involves Factorials.
    
    (fill-in)
    
    I will not be studying Pascal's Triangle in depth, as Don Knuth notes in the next section. 
    There are a couple of thousand known patterns in binomial numbers and Pascal's Triangle.
    
    I will be studying the field of number triangles.
    
    I may need two meetings, to share most everything with you?
    
    I have tools ranging from an explanation of how created, then a diagram, then a Windows 
    application, plus an online dataset available to you.

    % ------------------------------------------------------------------
     
    \subsection{Don Knuth Quote}
    \label{sec:12}

    A quote on Pascal's Triangle and Binomial numbers --- discoveries.
    
    This is from page 54 of the 3rd Edition (most recent, as of 2025) TAOCP, Vol. Fundamental 
    Algorithms.
    
    \begin{quote}
    \textit{Binomial coefficients satisfy literally thousands of identities, and for centuries 
    their amazing properties have been continually explored.}
    
    \textit{In fact, there are so many relations present that when someone finds a new identity, 
    not many people get excited about it any more, except for the discoverer.}
    \end{quote}
    
    Persuade you that there is still interesting material in the subject. There are many number 
    triangles, with different patterns and types.
    
    Even though Don Knuth's quote is discouraging. I understand; mathematics is vast. It is 
    unlikely to find anything new. Yet, I have done this.
    
    I aim to investigate patterns and structural symmetries in number triangles and determine 
    whether these patterns can be connected to integer sequences in the OEIS database.
 
     % ------------------------------------------------------------------
     
    \subsection{Pascal's Triangle}
    \label{sec:13}

    Pascal's Triangle has been realized, found, discovered, and rediscovered several times. It 
    was known in China and the Islamic world before Pascal studied its properties and published 
    it.
    
    Blaise Pascal collected results from several mathematicians who had discovered it. Then, he 
    added more findings and realized their connection to Probability and Combinatorics.
    
        \subsubsection{RSD}    
               
        \begin{itemize}
            \item \textit{Including Rising Shallow Diagonal Diagram (RSD)}
        \end{itemize}
        
        What are they? Why is it important? How to calculate them?
        
        I have a diagram, which shows how they are arranged. It starts on the left-hand side of a 
        number triangle and slopes upwards.
        
        Is there a Binomial formula? Showing the calculation of an RSD sum in Pascal's Triangle?
        
        (fill-in)
        
        What are the applications of RSD values? Starting with it being an actual integer 
        sequence.
        
\begin{enumerate}
    \item They are building blocks for infinite series. You can derive an approximate series from 
    an exact integer sequence. Then, apply it to a numerical calculation algorithm.
    \item Data structures and runtime analyses (big-O notation) in Computer science, often 
    produce connections to integer sequences.
    \item Numbers surround us. Finance, Construction patterns, or scheduling routines all involve 
    simple integer patterns.
\end{enumerate}
        
        \subsubsection{Number Triangle Format}
        
        \begin{itemize}
            \item \textit{Difference of Centered \& Left-Justified Number Triangles}
        \end{itemize}
        
        The following diagram, shows the difference between a centered number triangle and a 
        left-justified number triangle.
        
        I find the left-justified number triangle easier to read, in several ways. However, all 
        my number triangle diagrams are centered.
        
        \subsubsection{Simplex Numbers}
        
        \begin{itemize}
            \item \textit{Simplex Numbers \& Simplices}
        \end{itemize}
        
        What are the Simplex numbers? Refers to a family of "Figurate numbers" when generalizing 
        the polygonal numbers.
        
        \begin{enumerate}
            \item Natural numbers
            \item Triangular numbers
            \item Tetrahedral numbers
            \item Pentalope numbers
            \item and more.
        \end{enumerate}
        
        Are there patterns to them, in their formulas? Yes.
        
        (fill-in)
        
        Gradually, it becomes x times, (x+1) times, (x+2) times, etc.
        
        Additional division by 1, 1 times 2, times 3, times 4, ..
        
        They are very special numbers in Pascal's Triangle. However, we don't see them often, 
        beyond the Pentalope simplex numbers.
        
        \subsubsection{Geometric Relations}
        
        \begin{itemize}
            \item 0D: point.
            \item 1D: line segment.
            \item 2D: triangle.
            \item 3D: tetrahedron
            \item 4D: called 4-simplex, or 5-cell pentachoron, or pentatope.
        \end{itemize}

    % ------------------------------------------------------------------

     
    \subsection{Twice-Filled Pascal's Triangle}
    \label{sec:14}

    % ------------------------------------------------------------------

     
    \subsection{Rudimentary Topics}
    \label{sec:15}

    % ------------------------------------------------------------------

     
    \subsection{Teardrop Number Triangle}
    \label{sec:16}

    % ------------------------------------------------------------------

     
    \subsection{Golden Triangle}
    \label{sec:17}

    % ------------------------------------------------------------------

     
    \subsection{NTA2 Application}
    \label{sec:18}

    % ------------------------------------------------------------------
     
    \subsection{Vocabulary}
    \label{sec:19}

    

% ------------------------------------------------------------------

% ============================================================    
    
\end{document}



